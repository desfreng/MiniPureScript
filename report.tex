\documentclass[french, 12pt]{article}
\usepackage[a4paper, top=1cm, bottom=2cm, left=1cm, right=1cm]{geometry}

\usepackage{lmodern}
\usepackage[T1]{fontenc}
\usepackage[french]{babel}
\usepackage[indent]{parskip}
\usepackage[bottom]{footmisc}
\usepackage{amsmath}
\usepackage{minted}
\usepackage{calc}
\usepackage{enumitem}
\usepackage{hyperref}

\newcommand{\ps}{\textsf{PureScript}}
\newcommand{\pps}{\textsf{Petit Purescript}}

\title{{\large Projet de Compilation}\\\pps}
\author{Gabriel Desfrene}
\begin{document}
\maketitle

Le compilateur \texttt{ppurse} ici présenté, traite et supporte la totalité du
sujet. De plus, ce compilateur supporte quelques extensions :
\begin{itemize}
    \item Le typage des fonctions peut être rendu plus permissif, en autorisant
          le filtrage sur tous les arguments à l'aide de l'option
          \texttt{---permissive}.
    \item Les expressions du type \texttt{Effect a} sont compilée comme des
          clôtures afin d'imiter au plus le comportement du compilateur \ps.
    \item Les constantes entières négatives sont autorisées dans les motifs de
          filtrage.
\end{itemize}

\section*{Analyse lexicale et syntaxique}
Aucune difficulté majeure a été rencontrée dans cette partie. La grammaire a
légèrement été modifiée afin de supprimer des conflits
\emph{réduction-réduction}. Un nouveau symbole a été introduit,
\texttt{cio\_decl}\footnote{\it Class or Instance declaration.} afin que les
références à des instances ou des classes de type soient différentiée des
références à des symboles de types. Enfin, la règle
$\left\langle\textsl{patarg}\right\rangle$ a été modifié afin d'accepter une
constante entière négative.

\section*{Analyse sémantique, compilation des filtrages et résolution des instances}
La déclaration des types est réalisée de la manière suivante :
\begin{minted}{ocaml}
type ttyp =
  | TQuantifiedVar of QTypeVar.t
  | TVar of {id: TypeVar.t; mutable def: ttyp option}
  | TSymbol of Symbol.t * ttyp list
\end{minted}

Les variables de types sont alors séparés en deux possibilités :
\begin{itemize}
    \item Les variables de type universellement quantifié : \verb|TQuantifiedVar|.
          Ce sont les variables introduites lors des déclarations de fonctions,
          de classes de types, d'instances ou de symbole de types.
    \item Les variables de type unifiable : \verb|TVar|. Ce sont des variables
          qui doivent être unifiée après leur introduction lors des appels de
          fonction ou l'application de constructeurs.
\end{itemize}

Lors de la vérification de l'exhaustivité d'un filtrage, ce dernier est
transformé selon le type de l'expression filtrée :
\begin{itemize}
    \item Les filtrages sur des expressions booléennes sont transformée en des
          branchements conditionnels.
    \item Les filtrages sur les entiers sont distingués des filtrages
          sur les chaînes de caractères pour faciliter leur compilation dans la
          suite. Ils sont donc représentés sous la forme de dictionnaire ayant
          pour clef des entiers ou des chaînes de caractères. Cette solution
          n'est pas la plus plaisante, car presque redondante, mais elle permet
          d'assurer l'uniformité du type des constantes filtrant l'expression.
    \item Les filtrages sur des constructeurs de type sont transcrits par un
          dictionnaire associant chaque continuation à un constructeur.
\end{itemize}

Les instances de types sont résolue après l'analyse sémantique de la fonction
dans laquelle la résolution prend lieu. On procède de cette manière afin
d'accepter, comme \ps, le code du test \mbox{\texttt{lazy-inst.purs}}.

Le fichier \texttt{TypedAst.ml} décrit la représentation d'un programme après
l'analyse sémantique.

\section*{Simplification du programme}
Une passe de simplification est effectuée entre le typage et la suite de la
compilation. Cette étape permet de propager les constantes et de procéder à de
procéder à de nombreuses substitutions sur les variables introduites lors de la
compilation des filtrages et des fonctions. Lors de cette étape, les filtrages
sur les constantes (entiers ou chaînes de caractères) sont transformés en des
arbres binaires de recherche. De plus, certaines expressions constantes sont, en
partie, évaluées par le compilateur. Enfin, lors de cette étape, les opérateurs
binaires sont classés dans plusieurs catégories selon le type de leurs arguments
et de leur résultat.

Le fichier \texttt{SympAst.ml} décrit la représentation d'un programme après
la passe de simplification.

\section*{Allocation des variables locales}
La passe d'allocation des variables locales calcule la taille du tableau
d'activation de chaque fonction. De plus, lors de cette passe, les variables
locales sont positionnées dans la mémoire (par rapport à \texttt{\%rbp}). C'est
également lors de cette étape que les blocs \texttt{do} correspondant à la
construction d'une clôture à partir d'autres clôtures sont traités. On calcule
alors l'ensemble des variables libres et des instances locales (celles données
en argument à une fonction) utilisées au sein des blocs \texttt{do}. Puis,
le code correspondant au bloc \texttt{do} est placé dans une nouvelle fonction
accédant à ses arguments via la clôture toujours placée dans registre
\texttt{\%r12}. On fera attention à sauvegarder cette clôture lors de l'appel à
une clôture dans une clôture.

Le fichier \texttt{AllocAst.ml} décrit la représentation d'un programme après
la passe d'allocation des variables et le traitement des clôtures introduites
par les blocs \texttt{do}.

\section*{Compilation vers x86\_64}
On respecte le schéma de compilation proposé dans le sujet. On impose également
que lors d'un appel à une clôture, cette dernière est placée dans le registre
\texttt{\%r12}. Lors de la construction de nouvelles instances par un schéma de
classe de type, on se sert de l'instance construite par le schéma afin de
stocker les instances dont elle nécessite. Celles-ci sont placées à la suite des
pointeurs vers le code des fonctions déclarées dans la classe de type. Enfin,
afin que les fonctions implémentées dans un schéma d'instance de type aient
accès aux instances dont elles font usage, on place toujours l'instance dans
laquelle elles ont été déclarée sur le tas, avant tous les arguments\footnote{
    Comme une fonction classique.}

Les fonctions de la bibliothèque standard du C utilisées sont encapsulés dans
des fonctions respectant la convention d'appel proposée dans le sujet. Ces
fonctions ainsi que les fonctions prédéfinies sont toujours ajoutées au code
assembleur produit. Quelques remarques sur ces fonctions :
\begin{itemize}
    \item La fonction \texttt{pure} est compilée comme la fonction retournant
          une clôture de la fonction identité.
    \item La fonction \texttt{log} est compilée comme la fonction retournant une
          clôture de la fonction affichant du texte à sur la console.
    \item La fonction \texttt{mod} n'a pas été \textit{inline} au sein du code
          produit. En effet, cette fonction émule le comportement de la fonction
          \texttt{mod} de \ps, et nécessite donc plus qu'une poignée
          d'instruction.
    \item Une fonction \texttt{div} est introduite afin de simuler, de la même
          manière que \texttt{mod}, la division de \ps.
\end{itemize}

On remarquera d'ailleurs qu'en \ps :
\[
    \forall x.~ \texttt{mod}\left(x, 0\right) = 0 \text{ et }
    \texttt{div}\left(x, 0\right) = 0
\]

\section*{Remarques}
Ce projet fût très intéressant et pleins de rebondissement, en particulier la
partie de production de code. Quelques améliorations pourraient cependant être
ajoutés au code :
\begin{itemize}
    \item Une bonne refactorisation du code, en particulier lors de l'analyse
          sémantique.
    \item Des meilleurs messages d'erreurs lors d'un filtrage non exhaustif, en
          énumérant les cas non traités.
    \item Réaliser une allocation des registres afin de grandement diminuer le
          nombre d'opérations sur la pile
    \item Réaliser une meilleure sélection des instructions.
\end{itemize}

Quelques tests ont été ajoutés afin de vérifier le comportement du code produit
par le compilateur. Parmi ceux-ci, on notera :
\begin{itemize}
    \item \texttt{pi-with-bbp.purs} : Ce test approxime $\pi$ à l'aide de la
          formule de \textsc{Bailey}-\textsc{Borwein}-\textsc{Plouffe}%
          \footnote{Voir : \url{https://fr.wikipedia.org/wiki/Formule_BBP}}
    \item \texttt{modulo.purs} : Ce test (généré) vérifie le comportement de la
          fonction \texttt{mod}
    \item \texttt{division.purs} : Même qu'au-dessus pour la division.
    \item \texttt{lazy-inst.purs} : Ce test permet de vérifier la résolution des
          instances.
    \item \texttt{lazy-bool-op.purs} : Ce test permet de vérifier que les
          opérations booléennes sont bien paresseuses à l'aide d'une boucle infinie.
    \item \texttt{fibonacci.purs} : Ce test calcule quelques termes de la suite
          de Fibonacci à l'aide d'une liste.
    \item \texttt{evil\_effects.purs} : Ce test vérifie que la compilation des
          effets est similaire à celle dans \ps.
    \item \texttt{effect\_and\_inst.purs} : Ce test vérifie la compilation des
          schémas d'instances.
    \item \texttt{case4.purs} : Ce test vérifie la compilation du filtrage sur
          les constantes.
\end{itemize}
\end{document}
