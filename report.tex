\documentclass[french, 12pt]{article}
\usepackage[a4paper, top=1cm, bottom=2cm, left=1cm, right=1cm]{geometry}

\usepackage{lmodern}
\usepackage[T1]{fontenc}
\usepackage[french]{babel}
\usepackage[indent]{parskip}
\usepackage[bottom]{footmisc}
\usepackage{minted}
\usepackage{calc}
\usepackage{enumitem}
\usepackage{hyperref}

\newcommand{\ps}{\textsf{Purescript}}
\newcommand{\pps}{\textsf{Petit Purescript}}

\title{{\large Projet de Compilation}\\\pps}
\author{Gabriel Desfrene}
\begin{document}
\maketitle

\section*{Organisation}
Ce projet utilise \textsf{dune}\footnote{\url{https://dune.build/}} comme moteur
de production.
Le code du compilateur \pps{} est séparé en deux parties :
\begin{itemize}
    \item Une bibliothèque \textsf{PetitPureScript}, dont les sources se trouvent
          dans le répertoire \texttt{lib}.
    \item Le compilateur \textsf{ppurse}, dont les sources sont localisées dans
          le répertoire \texttt{bin}.
\end{itemize}

\section*{Compilateur \textsf{ppurse}}
Ce programme accepte les options suivantes :
\begin{description}[font=\normalfont, leftmargin=!, labelwidth=\widthof{\texttt{----permissive}}]
    \item[\texttt{----help}] affiche l'aide pour l'utilisation de ce compilateur.
    \item[\texttt{----parse-only}] stoppe le compilateur après l'analyse
        syntaxique du fichier d'entrée.
    \item[\texttt{----type-only}] stoppe le compilateur après l'analyse
        sémantique du fichier d'entrée.
    \item[\texttt{----permissive}] autorise la définition de fonction avec des
        motifs de filtrages quelconques sur tous les arguments.
\end{description}

Le compilateur termine avec le code de sortie $0$ lorsque aucune erreur n'est
remarquée. Un code de sortie $1$ signale une erreur lors de l'analyse du fichier
d'entrée. Un code de sortie $2$ marque une erreur interne du compilateur.

\section*{Bibliothèque \textsf{PetitPureScript}}
Cette bibliothèque s'occupe d'implémenter et de définir toutes les structures
nécessaires à l'implémentation du compilateur \textsf{ppurse}.

\subsection*{Analyse lexicale}

L'analyse lexicale est opérée à l'aide de l'outil \textsf{ocamllex}%
\footnote{\url{https://v2.ocaml.org/manual/lexyacc.html}} au sein du fichier
\verb|Lexer.mll|. Aucun problème n'a été rencontré lors de cette partie.
Dans la suite de ce rapport ainsi qu'au sein du code de ce projet, les
lexèmes renvoyés par l'analyseur lexical sont appelés \emph{pretokens}. Ce
sont des lexèmes étiquetés par une position.

L'ajout des lexèmes factices \verb|{|, \verb|}| et \verb|;| est réalisé au sein
du fichier \verb|PostLexer.ml|. Ce fichier implémente l'algorithme proposé dans
le sujet. Une fonction \verb|last_pos| est implémenté afin d'obtenir la position
du dernier lexème retourné.

\subsection*{Analyse syntaxique}
L'analyse syntaxique est opérée à l'aide de l'outil \textsf{Menhir}%
\footnote{\url{http://gallium.inria.fr/~fpottier/menhir}}. Le fichier
\verb|Tokens.mly| défini les lexèmes ainsi que leurs priorités. Le fichier
\verb|Parser.mly| s'occupe de construire l'arbre de syntaxe abstraite (AST)
défini au sein du fichier \verb|Ast.ml|.

La grammaire des types a été légèrement modifiée afin de supprimer des conflits
\emph{reduce-reduce}. De plus, puisque les classes de type et les instances,
qui partagent la même grammaire, ont un sens différent des types, un nouveau
symbole a été introduit : \verb|cio_decl|%
\footnote{\it Class or Instance declaration.}. On retrouve cette différence de
type au sein de l'AST.

Mis à part ces quelques remarques, aucune difficulté majeure n'a été rencontrée.

\subsection*{Analyse sémantique}
L'analyse sémantique est opérée au sein des fichiers du dossier \verb|lib/typing|.
Lors de cette analyse, l'arbre de syntaxe abstraite produit à l'étape précédente
est transformé en un nouveau, défini au sein du fichier \verb|TAst.ml|.

La déclaration des types est réalisée de la manière suivante :
\begin{minted}{ocaml}
type ttyp =
  | TQuantifiedVar of TQVar.t
  | TVar of { id : TVar.t; mutable def : ttyp option }
  | TSymbol of string * ttyp list
\end{minted}

Les variables de types sont séparés en deux possibilités :
\begin{itemize}
    \item Les variables de type universellement quantifiées : \verb|TQuantifiedVar|.
          Ce sont les variables introduites lors des déclarations de fonctions,
          de classes de types, d'instances ou de symbole de types.
    \item Les variables de type unifiable : \verb|TVar|. Ce sont des variables
          qui doivent être unifiée après leur introduction lors des appels de
          fonction ou l'application de constructeurs.
\end{itemize}

L'analyse sémantique et le regroupement des déclarations est effectuée au sein
du fichier \verb|Typing.ml|. L'analyse des expressions est effectuée dans le
fichier \verb|ExpressionTyping.ml|. Les motifs de filtrages sont analysés et
typés dans le fichier \verb|PatternTyping.ml|. Le fichier \verb|CommonTyping.ml|
regroupe des déclarations utilisées par plusieurs fichiers.

On remarque qu'en \ps{} le code de la \autoref{fig:ex} est correct. Alors que
la variable \verb|x| est de type \verb|List 'a| lors de l'appel de la fonction
\verb|log| à la ligne $15$, ce code est accepté. En effet, le type de la
variable \verb|x| est résolu à la ligne suivante, \verb|x| est de type
\verb|List Int|. Ce code illustre que la résolution des instances est réalisée
après l'analyse de l'expression des fonctions. Cette résolution est effectuée
au sein du fichier \verb|ResolveInstance.ml|.

Le filtrage est compilé en un \emph{case} sur tous les constructeurs possibles
du motif au sein du fichier \verb|CompileCase.ml|. Enfin, le fichier
\verb|TypingError.ml| contient le code nécessaire à la production des erreurs
rencontrée lors de l'analyse sémantique.

\begin{figure}[H]
    \begin{minted}[fontsize=\footnotesize,linenos]{haskell}
module Main where

import Prelude
import Effect
import Effect.Console

data List a = Nil | Cons a (List a)

instance Show a => Show (List a) where
    show Nil = "[]"
    show (Cons hd tl) = show hd <> "::" <> show tl

main::Effect Unit
main = let x = Nil in do
    log (show x)    -- x est de type List 'a
    (let y = Cons 1 x in log (show y))    -- qui est ici résolu en List Int
    log (show y)
    \end{minted}
    \caption{Exemple \ps{}.}
    \label{fig:ex}
\end{figure}
\end{document}
